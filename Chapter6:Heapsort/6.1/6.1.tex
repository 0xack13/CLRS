% XeLaTeX can use any Mac OS X font. See the setromanfont command below.
% Input to XeLaTeX is full Unicode, so Unicode characters can be typed directly into the source.

% The next lines tell TeXShop to typeset with xelatex, and to open and save the source with Unicode encoding.

%!TEX TS-program = xelatex
%!TEX encoding = UTF-8 Unicode

\documentclass[12pt]{article}
\usepackage{geometry}                % See geometry.pdf to learn the layout options. There are lots.
\geometry{letterpaper}                   % ... or a4paper or a5paper or ... 
%\geometry{landscape}                % Activate for for rotated page geometry
%\usepackage[parfill]{parskip}    % Activate to begin paragraphs with an empty line rather than an indent
\usepackage{graphicx}
\usepackage{amssymb}
\usepackage[colorlinks,linkcolor=black,anchorcolor=blue,citecolor=green]{hyperref}

% Will Robertson's fontspec.sty can be used to simplify font choices.
% To experiment, open /Applications/Font Book to examine the fonts provided on Mac OS X,
% and change "Hoefler Text" to any of these choices.

\usepackage{fontspec,xltxtra,xunicode}
\defaultfontfeatures{Mapping=tex-text}
\setromanfont[Mapping=tex-text]{Hoefler Text}
\setsansfont[Scale=MatchLowercase,Mapping=tex-text]{Gill Sans}
\setmonofont[Scale=MatchLowercase]{Andale Mono}

\title{算法导论习题6.1}
\author{Louis1992  \\ 有需要可以\href{mailto:zhenchaogan@hotmail.com}{联系我} 
\\ 我的github\href{https://github.com/gzc}{欢迎大家帮我完成算法导论}   \\ \href{https://gzc.github.io}{我的博客}}
%\date{}                                           % Activate to display a given date or no date

\usepackage[slantfont,boldfont]{xeCJK}
\usepackage{xcolor}
\setCJKmainfont{SimSun}
\setCJKfamilyfont{song}{SimSun}


\begin{document}
\maketitle

% For many users, the previous commands will be enough.
% If you want to directly input Unicode, add an Input Menu or Keyboard to the menu bar 
% using the International Panel in System Preferences.
% Unicode must be typeset using a font containing the appropriate characters.
% Remove the comment signs below for examples.

% \newfontfamily{\A}{Geeza Pro}\sqrt[n]{}
% \newfontfamily{\H}[Scale=0.9]{Lucida Grande}
% \newfontfamily{\J}[Scale=0.85]{Osaka}

% Here are some multilingual Unicode fonts: this is Arabic text: {\A السلام عليكم}, this is Hebrew: {\H שלום}, 
% and here's some Japanese: {\J 今日は}.

\noindent Exercises 6.1-1 What are the minimum and maximum numbers of elements in a heap of height h? \\
最多就是一颗很完美的二叉树,是 $2^{h+1}-1$ ;
最少的话最后一层只有一个,是 $2^{h}$ 
\\ 
\\

\noindent Exercises 6.1-2
Show that an n-element heap has height $\llcorner\lg{n}\lrcorner$ \\
直接利用第一题的结论:$2^{h+1}-1\geq x \geq 2^{h} \rightrightarrows  \lg{x} \geq h \geq \lg{(x+1)}-1 $ \\
所以 h = $\llcorner\lg{n}\lrcorner$
\\
\\

\noindent Exercises 6.1-3
Show that in any subtree of a max-heap, the root of the subtree contains the largest value occurring anywhere in that subtree. \\
这就是最大堆的性质!
\\
\\

\noindent Exercises 6.1-4
Where in a max-heap might the smallest element reside, assuming that all elements are distinct? \\
肯定是在叶子节点
\\
\\

\noindent Exercises 6.1-5
Is an array that is in sorted order a min-heap? \\
没有说明是递增数组还是递减数组,所以不一定
\\
\\

\noindent Exercises 6.1-6
Is the sequence [23, 17, 14, 6, 13, 10, 1, 5, 7, 12] a max-heap? \\
不是,7 > 6
\\
\\

\noindent Exercises 6.1-7
Show that, with the array representation for storing an n-element heap, the leaves are the nodes indexed by $\llcorner{n/2}\lrcorner$ + 1, $\llcorner{n/2}\lrcorner$ + 2, ... , n. \\
也是很简单的性质


\end{document}  